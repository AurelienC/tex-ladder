\documentclass[]{article}
\usepackage{tikz}
\usepackage[utf8]{inputenc} % Définition langue
\usepackage[T1]{fontenc} % Définition langue
\usepackage[frenchb]{babel} % Définition langue
\usepackage{Ladder}






\begin{document}

\begin{figure}
\
\begin{tikzpicture}
\LadderLine % Début d'une ligne de ladder

\LadderNO{bla}{I1.0} % Contact NO

\startParallel % Il y aura le début d'une section parrallèle après le contact NO

\LadderNC{bli}{M0} % Contact NC


\setParallel % La fin de la section parrallèle se situe après le contact NC, on définit ensuite cette section parallèle
\LadderNO{blou}{I1.0}
\LadderNO{blo}{I1.0}
\unsetParallel % Fin de la définition de la section parrallèle

\LadderB{blu}{M1} % On met une "bobine"

\LadderNO[P]{ble}{I1.0}

\startParallel

\LadderB[R]{bleu}{M1} % Bobine "R" reset

\setParallel
\LadderB[S]{bleu}{M1}
\unsetParallel


% Nouvelle ligne %
\LadderLine

\LadderNO{bla}{I1.0}

\startParallel

\LadderNC{bli}{M0}
\LadderNO[P]{blou}{I1.0}

\setParallel
\LadderNO{blo}{I1.0}
\unsetParallel

\LadderB{blu}{M1}

\LadderNO{ble}{I1.0}

\startParallel

\LadderB{bleu}{M1}{R}

\setParallel
\LadderNO{ble}{I1.0}
\LadderB[S]{bleu}{M1}
\unsetParallel

\end{tikzpicture}
\caption{}
\end{figure}

\end{document}
